% Options for packages loaded elsewhere
\PassOptionsToPackage{unicode}{hyperref}
\PassOptionsToPackage{hyphens}{url}
%
\documentclass[
]{article}
\usepackage{lmodern}
\usepackage{amssymb,amsmath}
\usepackage{ifxetex,ifluatex}
\ifnum 0\ifxetex 1\fi\ifluatex 1\fi=0 % if pdftex
  \usepackage[T1]{fontenc}
  \usepackage[utf8]{inputenc}
  \usepackage{textcomp} % provide euro and other symbols
\else % if luatex or xetex
  \usepackage{unicode-math}
  \defaultfontfeatures{Scale=MatchLowercase}
  \defaultfontfeatures[\rmfamily]{Ligatures=TeX,Scale=1}
\fi
% Use upquote if available, for straight quotes in verbatim environments
\IfFileExists{upquote.sty}{\usepackage{upquote}}{}
\IfFileExists{microtype.sty}{% use microtype if available
  \usepackage[]{microtype}
  \UseMicrotypeSet[protrusion]{basicmath} % disable protrusion for tt fonts
}{}
\makeatletter
\@ifundefined{KOMAClassName}{% if non-KOMA class
  \IfFileExists{parskip.sty}{%
    \usepackage{parskip}
  }{% else
    \setlength{\parindent}{0pt}
    \setlength{\parskip}{6pt plus 2pt minus 1pt}}
}{% if KOMA class
  \KOMAoptions{parskip=half}}
\makeatother
\usepackage{xcolor}
\IfFileExists{xurl.sty}{\usepackage{xurl}}{} % add URL line breaks if available
\IfFileExists{bookmark.sty}{\usepackage{bookmark}}{\usepackage{hyperref}}
\hypersetup{
  pdftitle={Assignment 1},
  pdfauthor={Landon Buell},
  hidelinks,
  pdfcreator={LaTeX via pandoc}}
\urlstyle{same} % disable monospaced font for URLs
\usepackage[margin=1in]{geometry}
\usepackage{color}
\usepackage{fancyvrb}
\newcommand{\VerbBar}{|}
\newcommand{\VERB}{\Verb[commandchars=\\\{\}]}
\DefineVerbatimEnvironment{Highlighting}{Verbatim}{commandchars=\\\{\}}
% Add ',fontsize=\small' for more characters per line
\usepackage{framed}
\definecolor{shadecolor}{RGB}{248,248,248}
\newenvironment{Shaded}{\begin{snugshade}}{\end{snugshade}}
\newcommand{\AlertTok}[1]{\textcolor[rgb]{0.94,0.16,0.16}{#1}}
\newcommand{\AnnotationTok}[1]{\textcolor[rgb]{0.56,0.35,0.01}{\textbf{\textit{#1}}}}
\newcommand{\AttributeTok}[1]{\textcolor[rgb]{0.77,0.63,0.00}{#1}}
\newcommand{\BaseNTok}[1]{\textcolor[rgb]{0.00,0.00,0.81}{#1}}
\newcommand{\BuiltInTok}[1]{#1}
\newcommand{\CharTok}[1]{\textcolor[rgb]{0.31,0.60,0.02}{#1}}
\newcommand{\CommentTok}[1]{\textcolor[rgb]{0.56,0.35,0.01}{\textit{#1}}}
\newcommand{\CommentVarTok}[1]{\textcolor[rgb]{0.56,0.35,0.01}{\textbf{\textit{#1}}}}
\newcommand{\ConstantTok}[1]{\textcolor[rgb]{0.00,0.00,0.00}{#1}}
\newcommand{\ControlFlowTok}[1]{\textcolor[rgb]{0.13,0.29,0.53}{\textbf{#1}}}
\newcommand{\DataTypeTok}[1]{\textcolor[rgb]{0.13,0.29,0.53}{#1}}
\newcommand{\DecValTok}[1]{\textcolor[rgb]{0.00,0.00,0.81}{#1}}
\newcommand{\DocumentationTok}[1]{\textcolor[rgb]{0.56,0.35,0.01}{\textbf{\textit{#1}}}}
\newcommand{\ErrorTok}[1]{\textcolor[rgb]{0.64,0.00,0.00}{\textbf{#1}}}
\newcommand{\ExtensionTok}[1]{#1}
\newcommand{\FloatTok}[1]{\textcolor[rgb]{0.00,0.00,0.81}{#1}}
\newcommand{\FunctionTok}[1]{\textcolor[rgb]{0.00,0.00,0.00}{#1}}
\newcommand{\ImportTok}[1]{#1}
\newcommand{\InformationTok}[1]{\textcolor[rgb]{0.56,0.35,0.01}{\textbf{\textit{#1}}}}
\newcommand{\KeywordTok}[1]{\textcolor[rgb]{0.13,0.29,0.53}{\textbf{#1}}}
\newcommand{\NormalTok}[1]{#1}
\newcommand{\OperatorTok}[1]{\textcolor[rgb]{0.81,0.36,0.00}{\textbf{#1}}}
\newcommand{\OtherTok}[1]{\textcolor[rgb]{0.56,0.35,0.01}{#1}}
\newcommand{\PreprocessorTok}[1]{\textcolor[rgb]{0.56,0.35,0.01}{\textit{#1}}}
\newcommand{\RegionMarkerTok}[1]{#1}
\newcommand{\SpecialCharTok}[1]{\textcolor[rgb]{0.00,0.00,0.00}{#1}}
\newcommand{\SpecialStringTok}[1]{\textcolor[rgb]{0.31,0.60,0.02}{#1}}
\newcommand{\StringTok}[1]{\textcolor[rgb]{0.31,0.60,0.02}{#1}}
\newcommand{\VariableTok}[1]{\textcolor[rgb]{0.00,0.00,0.00}{#1}}
\newcommand{\VerbatimStringTok}[1]{\textcolor[rgb]{0.31,0.60,0.02}{#1}}
\newcommand{\WarningTok}[1]{\textcolor[rgb]{0.56,0.35,0.01}{\textbf{\textit{#1}}}}
\usepackage{graphicx,grffile}
\makeatletter
\def\maxwidth{\ifdim\Gin@nat@width>\linewidth\linewidth\else\Gin@nat@width\fi}
\def\maxheight{\ifdim\Gin@nat@height>\textheight\textheight\else\Gin@nat@height\fi}
\makeatother
% Scale images if necessary, so that they will not overflow the page
% margins by default, and it is still possible to overwrite the defaults
% using explicit options in \includegraphics[width, height, ...]{}
\setkeys{Gin}{width=\maxwidth,height=\maxheight,keepaspectratio}
% Set default figure placement to htbp
\makeatletter
\def\fps@figure{htbp}
\makeatother
\setlength{\emergencystretch}{3em} % prevent overfull lines
\providecommand{\tightlist}{%
  \setlength{\itemsep}{0pt}\setlength{\parskip}{0pt}}
\setcounter{secnumdepth}{-\maxdimen} % remove section numbering

\title{Assignment 1}
\usepackage{etoolbox}
\makeatletter
\providecommand{\subtitle}[1]{% add subtitle to \maketitle
  \apptocmd{\@title}{\par {\large #1 \par}}{}{}
}
\makeatother
\subtitle{CS 750/850 Machine Learning}
\author{Landon Buell}
\date{29 January 2020}

\begin{document}
\maketitle

\begin{itemize}
\tightlist
\item
  \textbf{Due}: Monday 2/3 at 11:59PM
\item
  \textbf{Submisssion}: Turn in as a \textbf{PDF} and the \textbf{source
  code} (R,Rmd,py,ipynb) on \href{http://mycourses.unh.edu}{MyCourses}
\item
  \textbf{Questions}:
  \href{piazza.com/unh/spring2020/cs750cs850}{Piazza} and Office hours:
  \emph{Marek}: Wed 1:30-3:00pm, Soheil: Mon 2-4pm, Xihong: Thu
  1:30-3:30pm
\item
  \textbf{Extra credit}: Especially good questions or helpful answers on
  Piazza regarding the assignment earn up to 5 points extra credit
  towards the assignment grade.
\end{itemize}

The instructions are geared towards R users. The comments in {[}P:
xxx{]} are meant as hints to Python users.

Feel free to achieve the results using commands other than the ones
mentioned. R, especially, shines when it comes to processing and
visualization of structured data, but you would not be able to tell from
the ISL book. It uses the oldest and simplest (and ugliest) subset of R.
I recommend checking out \texttt{dplyr} for data processing {[}3,4{]}
and GGPlot {[}1,4{]} for plotting.

\hypertarget{problem-1-25}{%
\subsection{Problem 1 {[}25\%{]}}\label{problem-1-25}}

In this exercise you will create some simulated data and will fit simple
linear regression models to it. Make sure to use \texttt{set.seed(1)}
{[}P: \texttt{np.random.seed(1)}{]} prior to starting part (1) to ensure
consistent results.

\begin{enumerate}
\def\labelenumi{\arabic{enumi}.}
\tightlist
\item
  Using the \texttt{rnorm()} {[}P: \texttt{np.random.normal}{]}
  function, create a vector, \texttt{x}, containing \(100\) observations
  drawn from a \(\mathcal{N}(0, 3)\) distribution (Normal distribution
  with the mean \(0\) and the \textbf{standard deviation} \(\sqrt{3}\)).
  This represents a feature, \(X\).
\end{enumerate}

\begin{Shaded}
\begin{Highlighting}[]
\KeywordTok{set.seed}\NormalTok{(}\DecValTok{1}\NormalTok{)}
\NormalTok{x <-}\StringTok{ }\KeywordTok{rnorm}\NormalTok{(}\DataTypeTok{n=}\DecValTok{100}\NormalTok{,}\DataTypeTok{mean=}\DecValTok{0}\NormalTok{,}\DataTypeTok{sd=}\KeywordTok{sqrt}\NormalTok{(}\DecValTok{3}\NormalTok{))}
\end{Highlighting}
\end{Shaded}

\begin{enumerate}
\def\labelenumi{\arabic{enumi}.}
\setcounter{enumi}{1}
\tightlist
\item
  Using the \texttt{rnorm()} function, create a vector, \texttt{eps},
  containing \(100\) observations drawn from a \(\mathcal{N}(0, 0.5)\)
  distribution i.e.~a normal distribution with mean zero and standard
  deviation \(\sqrt{0.5}\).
\end{enumerate}

\begin{Shaded}
\begin{Highlighting}[]
\NormalTok{eps <-}\StringTok{ }\KeywordTok{rnorm}\NormalTok{(}\DataTypeTok{n=}\DecValTok{100}\NormalTok{,}\DataTypeTok{mean=}\DecValTok{0}\NormalTok{,}\DataTypeTok{sd=}\KeywordTok{sqrt}\NormalTok{(}\FloatTok{0.5}\NormalTok{))}
\end{Highlighting}
\end{Shaded}

\begin{enumerate}
\def\labelenumi{\arabic{enumi}.}
\setcounter{enumi}{2}
\tightlist
\item
  Using \texttt{x} and \texttt{eps}, generate a vector \texttt{y}
  according to the model \(Y\): \[ Y = -2 + 0.6 X + \epsilon \] What is
  the length (number of elements) of \texttt{y}? What are the values of
  \(\beta_0,\beta_1\) in the equation above (intercept and slope)?
\end{enumerate}

\begin{Shaded}
\begin{Highlighting}[]
\NormalTok{Y <-}\StringTok{ }\DecValTok{-2} \OperatorTok{+}\StringTok{ }\FloatTok{0.6}\OperatorTok{*}\NormalTok{x }\OperatorTok{+}\StringTok{ }\NormalTok{eps}
\KeywordTok{message}\NormalTok{(}\StringTok{"Elements in Y:"}\NormalTok{,}\KeywordTok{length}\NormalTok{(Y))}
\end{Highlighting}
\end{Shaded}

\begin{verbatim}
## Elements in Y:100
\end{verbatim}

In this model above, the intercept is given by \(\beta_0 = -2\) and the
slope is given by \(\beta_1 = +0.6\).

\begin{enumerate}
\def\labelenumi{\arabic{enumi}.}
\setcounter{enumi}{3}
\tightlist
\item
  Create a scatterplot displaying the relationship between \texttt{x}
  and \texttt{y}. Comment on what you observe. {[}P: see {[}2{]}{]}
\end{enumerate}

\begin{Shaded}
\begin{Highlighting}[]
\KeywordTok{plot}\NormalTok{(x,Y,}\DataTypeTok{main=}\StringTok{'Scatter Plot of X and Y'}\NormalTok{,}
     \DataTypeTok{xlab=}\StringTok{'X'}\NormalTok{,}\DataTypeTok{ylab=}\StringTok{'Y'}\NormalTok{)}
\end{Highlighting}
\end{Shaded}

\includegraphics{assignment1_files/figure-latex/unnamed-chunk-4-1.pdf}
The data seemes to be losely correlated, following a roughly linear
relationship. (As I would expect it to because of our linear model)

\begin{enumerate}
\def\labelenumi{\arabic{enumi}.}
\setcounter{enumi}{4}
\tightlist
\item
  Fit a least squares linear model to predict \texttt{y} using
  \texttt{x}. Comment on the model obtained. How do
  \(\hat{\beta}_0,\hat{\beta}_1\) compare to \(\beta_0,\beta_1\)?
\end{enumerate}

\begin{Shaded}
\begin{Highlighting}[]
\NormalTok{lin_fit <-}\StringTok{ }\KeywordTok{lm}\NormalTok{(Y}\OperatorTok{~}\NormalTok{x) }\CommentTok{# lm = linear model}
\NormalTok{lin_fit}
\end{Highlighting}
\end{Shaded}

\begin{verbatim}
## 
## Call:
## lm(formula = Y ~ x)
## 
## Coefficients:
## (Intercept)            x  
##     -2.0267       0.5996
\end{verbatim}

The regression line is a linear relationsip (1st degree polynomial) as
anticipated. It produces a fit such that the sum of the vertical
distances between each data point and the line has been minimized.\\
In our equation, we set the intercept and slope to be: \(\beta_0 = -2\)
and \(\beta_1 = +0.6\). repectivly. In the fit model, the linear model
found the approximations of the slpoe and intercept to be roughly:
\(\hat{\beta_0} = -2.0267\) and \(\hat{\beta_1} = +0.5996\) as shown by
the coeffefcients in the fit variable above.

\begin{enumerate}
\def\labelenumi{\arabic{enumi}.}
\setcounter{enumi}{5}
\tightlist
\item
  Display the least squares line on the scatterplot obtained in 4.
\end{enumerate}

\begin{Shaded}
\begin{Highlighting}[]
\KeywordTok{plot}\NormalTok{(x,Y,}\DataTypeTok{main=}\StringTok{'Scatter Plot of X and Y'}\NormalTok{,}
     \DataTypeTok{xlab=}\StringTok{'X'}\NormalTok{,}\DataTypeTok{ylab=}\StringTok{'Y'}\NormalTok{)}
\KeywordTok{abline}\NormalTok{(lin_fit)}
\end{Highlighting}
\end{Shaded}

\includegraphics{assignment1_files/figure-latex/unnamed-chunk-6-1.pdf}
7. Now fit a polynomial regression model that predicts \(y\) using \(x\)
and \(x^2\). Is there evidence that the quadratic term improves the
model fit? Explain your answer.

\begin{Shaded}
\begin{Highlighting}[]
\NormalTok{quad_fit =}\StringTok{ }\KeywordTok{lm}\NormalTok{(Y}\OperatorTok{~}\KeywordTok{poly}\NormalTok{(x,}\DecValTok{2}\NormalTok{))}
\NormalTok{quad_fit}
\end{Highlighting}
\end{Shaded}

\begin{verbatim}
## 
## Call:
## lm(formula = Y ~ poly(x, 2))
## 
## Coefficients:
## (Intercept)  poly(x, 2)1  poly(x, 2)2  
##     -1.9136       9.2809      -0.9504
\end{verbatim}

The quadratic model seems to be much further removed from the data set.
Ideally, I would have expected the intercept and slope coefficients to
remain fairly close to the repesctive linear model and then the
quadratic coefficient to be close to zero. TOo me this would indicate
that the quadratic term was not important, but still allow for the
retention of the properties of the linear model. Instead, the addition
of another polynomial term completely changes the best fit model to a
point where is differs greatly from the expected value. Now we have the
predictions: \(\hat{\beta}_0 = -19.136\), \(\hat{\beta}_1 = 9.2809\) and
\(\hat{\beta}_2 = -0.9504\).

\hypertarget{optional-problem-o1-30}{%
\subsection{Optional Problem O1
{[}30\%{]}}\label{optional-problem-o1-30}}

This problem can be substituted for Problem 1 above, for up to 5 points
extra credit. At most one of the problems 1 and O1 will be considered.

Read Chapter 1 and solve Exercises \emph{1.6} and \emph{1.10} in
{[}Bishop, C. M. (2006). Pattern Recognition and Machine Learning{]}.

\hypertarget{problem-2-25}{%
\subsection{Problem 2 {[}25\%{]}}\label{problem-2-25}}

Read through Section 2.3 in ISL. Load the \texttt{Auto} data set and
\emph{make sure to remove missing values from the data}. Then answer the
following questions:

\begin{Shaded}
\begin{Highlighting}[]
\CommentTok{# Below is from James, pg. 49}
\NormalTok{autodata <-}\StringTok{ }\KeywordTok{read.csv}\NormalTok{(}\DataTypeTok{file=}\StringTok{'Auto.csv'}\NormalTok{,}\DataTypeTok{header=}\NormalTok{T,}\DataTypeTok{na.strings=}\StringTok{"?"}\NormalTok{)}
\CommentTok{#fix(autodata) # I'm not sure what "fix()" does}
\KeywordTok{dim}\NormalTok{(autodata) }\CommentTok{# dimensions of array (nrows,ncols)}
\end{Highlighting}
\end{Shaded}

\begin{verbatim}
## [1] 397   9
\end{verbatim}

\begin{Shaded}
\begin{Highlighting}[]
\NormalTok{autodata =}\StringTok{ }\KeywordTok{na.omit}\NormalTok{(autodata) }\CommentTok{# eliminate "na's"}
\KeywordTok{dim}\NormalTok{(autodata) }\CommentTok{# dimensions of array (nrows,ncols)}
\end{Highlighting}
\end{Shaded}

\begin{verbatim}
## [1] 392   9
\end{verbatim}

\begin{Shaded}
\begin{Highlighting}[]
\CommentTok{# check variable names:}
\KeywordTok{names}\NormalTok{(autodata)}
\end{Highlighting}
\end{Shaded}

\begin{verbatim}
## [1] "mpg"          "cylinders"    "displacement" "horsepower"   "weight"      
## [6] "acceleration" "year"         "origin"       "name"
\end{verbatim}

\begin{enumerate}
\def\labelenumi{\arabic{enumi}.}
\tightlist
\item
  Which predictors are \emph{quantitative} and which ones are
  \emph{qualitative}?
\end{enumerate}

\begin{Shaded}
\begin{Highlighting}[]
\KeywordTok{summary}\NormalTok{(autodata)}
\end{Highlighting}
\end{Shaded}

\begin{verbatim}
##       mpg          cylinders      displacement     horsepower        weight    
##  Min.   : 9.00   Min.   :3.000   Min.   : 68.0   Min.   : 46.0   Min.   :1613  
##  1st Qu.:17.00   1st Qu.:4.000   1st Qu.:105.0   1st Qu.: 75.0   1st Qu.:2225  
##  Median :22.75   Median :4.000   Median :151.0   Median : 93.5   Median :2804  
##  Mean   :23.45   Mean   :5.472   Mean   :194.4   Mean   :104.5   Mean   :2978  
##  3rd Qu.:29.00   3rd Qu.:8.000   3rd Qu.:275.8   3rd Qu.:126.0   3rd Qu.:3615  
##  Max.   :46.60   Max.   :8.000   Max.   :455.0   Max.   :230.0   Max.   :5140  
##                                                                                
##   acceleration        year           origin                      name    
##  Min.   : 8.00   Min.   :70.00   Min.   :1.000   amc matador       :  5  
##  1st Qu.:13.78   1st Qu.:73.00   1st Qu.:1.000   ford pinto        :  5  
##  Median :15.50   Median :76.00   Median :1.000   toyota corolla    :  5  
##  Mean   :15.54   Mean   :75.98   Mean   :1.577   amc gremlin       :  4  
##  3rd Qu.:17.02   3rd Qu.:79.00   3rd Qu.:2.000   amc hornet        :  4  
##  Max.   :24.80   Max.   :82.00   Max.   :3.000   chevrolet chevette:  4  
##                                                  (Other)           :365
\end{verbatim}

The \textit{name}, \textit{origin}, and perhaps even the \textit{year}
are all quantitative. The \textit{MPG}, \textit{Cylinders},
\textit{Displacement}, \textit{horsepower}, \textit{weight} and
\textit{acceleration} categories are all made up of quantitative data.
2. What is the range, mean, and standard deviation of each predictor?
Use \texttt{range()} {[}\texttt{pandas.DataFrame.min} and
\texttt{max}{]} function.

\begin{Shaded}
\begin{Highlighting}[]
\KeywordTok{print}\NormalTok{(}\StringTok{"Mins & Maxes:"}\NormalTok{)}
\end{Highlighting}
\end{Shaded}

\begin{verbatim}
## [1] "Mins & Maxes:"
\end{verbatim}

\begin{Shaded}
\begin{Highlighting}[]
\CommentTok{# I tried to do this in a 'for loop' but I couldn't get it to work!}
\CommentTok{# I just hard-coded it, sorry (I'm a Python guy)}
\KeywordTok{cat}\NormalTok{(}\StringTok{"MPG:"}\NormalTok{,}\KeywordTok{range}\NormalTok{(autodata}\OperatorTok{$}\NormalTok{mpg),}\StringTok{"}\CharTok{\textbackslash{}n}\StringTok{"}\NormalTok{)}
\end{Highlighting}
\end{Shaded}

\begin{verbatim}
## MPG: 9 46.6
\end{verbatim}

\begin{Shaded}
\begin{Highlighting}[]
\KeywordTok{cat}\NormalTok{(}\StringTok{"Cylinders:"}\NormalTok{,}\KeywordTok{range}\NormalTok{(autodata}\OperatorTok{$}\NormalTok{cylinders),}\StringTok{"}\CharTok{\textbackslash{}n}\StringTok{"}\NormalTok{)}
\end{Highlighting}
\end{Shaded}

\begin{verbatim}
## Cylinders: 3 8
\end{verbatim}

\begin{Shaded}
\begin{Highlighting}[]
\KeywordTok{cat}\NormalTok{(}\StringTok{"Displacement:"}\NormalTok{,}\KeywordTok{range}\NormalTok{(autodata}\OperatorTok{$}\NormalTok{displacement),}\StringTok{"}\CharTok{\textbackslash{}n}\StringTok{"}\NormalTok{)}
\end{Highlighting}
\end{Shaded}

\begin{verbatim}
## Displacement: 68 455
\end{verbatim}

\begin{Shaded}
\begin{Highlighting}[]
\KeywordTok{cat}\NormalTok{(}\StringTok{"Horsepower:"}\NormalTok{,}\KeywordTok{range}\NormalTok{(autodata}\OperatorTok{$}\NormalTok{horsepower),}\StringTok{"}\CharTok{\textbackslash{}n}\StringTok{"}\NormalTok{)}
\end{Highlighting}
\end{Shaded}

\begin{verbatim}
## Horsepower: 46 230
\end{verbatim}

\begin{Shaded}
\begin{Highlighting}[]
\KeywordTok{cat}\NormalTok{(}\StringTok{"Weight:"}\NormalTok{,}\KeywordTok{range}\NormalTok{(autodata}\OperatorTok{$}\NormalTok{weight),}\StringTok{"}\CharTok{\textbackslash{}n}\StringTok{"}\NormalTok{)}
\end{Highlighting}
\end{Shaded}

\begin{verbatim}
## Weight: 1613 5140
\end{verbatim}

\begin{Shaded}
\begin{Highlighting}[]
\KeywordTok{cat}\NormalTok{(}\StringTok{"Acceleration:"}\NormalTok{,}\KeywordTok{range}\NormalTok{(autodata}\OperatorTok{$}\NormalTok{acceleration),}\StringTok{"}\CharTok{\textbackslash{}n}\StringTok{"}\NormalTok{)}
\end{Highlighting}
\end{Shaded}

\begin{verbatim}
## Acceleration: 8 24.8
\end{verbatim}

\begin{Shaded}
\begin{Highlighting}[]
\KeywordTok{cat}\NormalTok{(}\StringTok{"Year:"}\NormalTok{,}\KeywordTok{range}\NormalTok{(autodata}\OperatorTok{$}\NormalTok{year),}\StringTok{"}\CharTok{\textbackslash{}n}\StringTok{"}\NormalTok{)}
\end{Highlighting}
\end{Shaded}

\begin{verbatim}
## Year: 70 82
\end{verbatim}

\begin{Shaded}
\begin{Highlighting}[]
\KeywordTok{cat}\NormalTok{(}\StringTok{"Origin"}\NormalTok{,}\KeywordTok{range}\NormalTok{(autodata}\OperatorTok{$}\NormalTok{origin),}\StringTok{"}\CharTok{\textbackslash{}n}\StringTok{"}\NormalTok{)}
\end{Highlighting}
\end{Shaded}

\begin{verbatim}
## Origin 1 3
\end{verbatim}

\begin{enumerate}
\def\labelenumi{\arabic{enumi}.}
\setcounter{enumi}{2}
\tightlist
\item
  Investigate the predictors graphically using plots. Create plots
  highlighting relationships between predictors. See {[}1{]} for a
  ggplot cheatsheet.
\end{enumerate}

\begin{Shaded}
\begin{Highlighting}[]
\KeywordTok{library}\NormalTok{(ggplot2)}
\CommentTok{# ggplot(autodata,aes(x=mpg,y=year))}
\CommentTok{# how you you get ggplot to show points? It just produces an empty background!}

\KeywordTok{qplot}\NormalTok{(}\DataTypeTok{x=}\NormalTok{mpg,}\DataTypeTok{y=}\NormalTok{horsepower,}\DataTypeTok{data=}\NormalTok{autodata)}
\end{Highlighting}
\end{Shaded}

\includegraphics{assignment1_files/figure-latex/unnamed-chunk-11-1.pdf}

\begin{Shaded}
\begin{Highlighting}[]
\KeywordTok{qplot}\NormalTok{(}\DataTypeTok{x=}\NormalTok{mpg,}\DataTypeTok{y=}\NormalTok{weight,}\DataTypeTok{data=}\NormalTok{autodata)}
\end{Highlighting}
\end{Shaded}

\includegraphics{assignment1_files/figure-latex/unnamed-chunk-11-2.pdf}

\begin{Shaded}
\begin{Highlighting}[]
\KeywordTok{qplot}\NormalTok{(}\DataTypeTok{x=}\NormalTok{horsepower,}\DataTypeTok{y=}\NormalTok{weight,}\DataTypeTok{data=}\NormalTok{autodata)}
\end{Highlighting}
\end{Shaded}

\includegraphics{assignment1_files/figure-latex/unnamed-chunk-11-3.pdf}
4. Compute the matrix of correlations between variables using the
function \texttt{cor()} {[}P: \texttt{pandas.DataFrame.corr}{]}. Exclude
the \texttt{name} variable.\\
Matrix of Coreelations:

\begin{Shaded}
\begin{Highlighting}[]
\NormalTok{drops <-}\StringTok{ }\KeywordTok{c}\NormalTok{(}\StringTok{"name"}\NormalTok{)}
\NormalTok{auto2 <-}\StringTok{ }\NormalTok{autodata[ , }\OperatorTok{!}\NormalTok{(}\KeywordTok{names}\NormalTok{(autodata) }\OperatorTok\StringTok{ }\NormalTok{drops)]}
\CommentTok{# modifed this from Stack Overflow}
\KeywordTok{cor}\NormalTok{(auto2)}
\end{Highlighting}
\end{Shaded}

\begin{verbatim}
##                     mpg  cylinders displacement horsepower     weight
## mpg           1.0000000 -0.7776175   -0.8051269 -0.7784268 -0.8322442
## cylinders    -0.7776175  1.0000000    0.9508233  0.8429834  0.8975273
## displacement -0.8051269  0.9508233    1.0000000  0.8972570  0.9329944
## horsepower   -0.7784268  0.8429834    0.8972570  1.0000000  0.8645377
## weight       -0.8322442  0.8975273    0.9329944  0.8645377  1.0000000
## acceleration  0.4233285 -0.5046834   -0.5438005 -0.6891955 -0.4168392
## year          0.5805410 -0.3456474   -0.3698552 -0.4163615 -0.3091199
## origin        0.5652088 -0.5689316   -0.6145351 -0.4551715 -0.5850054
##              acceleration       year     origin
## mpg             0.4233285  0.5805410  0.5652088
## cylinders      -0.5046834 -0.3456474 -0.5689316
## displacement   -0.5438005 -0.3698552 -0.6145351
## horsepower     -0.6891955 -0.4163615 -0.4551715
## weight         -0.4168392 -0.3091199 -0.5850054
## acceleration    1.0000000  0.2903161  0.2127458
## year            0.2903161  1.0000000  0.1815277
## origin          0.2127458  0.1815277  1.0000000
\end{verbatim}

\begin{enumerate}
\def\labelenumi{\arabic{enumi}.}
\setcounter{enumi}{4}
\tightlist
\item
  Use the \texttt{lm()} function to perform a multiple linear regression
  with \texttt{mpg} as the response. {[}P: using \texttt{rpy} package is
  acceptable{]} Exclude \texttt{name} as a predictor, since it is
  qualitative. Briefly comment on the output: What is the relationship
  between the predictors? What does the coefficient for \texttt{year}
  variable suggest?\\
\end{enumerate}

\begin{Shaded}
\begin{Highlighting}[]
\NormalTok{fits <-}\StringTok{ }\KeywordTok{lm}\NormalTok{(mpg }\OperatorTok{~}\StringTok{ }\NormalTok{cylinders}\OperatorTok{+}\NormalTok{displacement }\OperatorTok{+}\StringTok{ }\NormalTok{horsepower}\OperatorTok{+}\NormalTok{weight}\OperatorTok{+}\NormalTok{acceleration}\OperatorTok{+}\NormalTok{year}\OperatorTok{+}\NormalTok{origin ,}\DataTypeTok{data=}\NormalTok{autodata)}
\KeywordTok{summary}\NormalTok{(fits)}
\end{Highlighting}
\end{Shaded}

\begin{verbatim}
## 
## Call:
## lm(formula = mpg ~ cylinders + displacement + horsepower + weight + 
##     acceleration + year + origin, data = autodata)
## 
## Residuals:
##     Min      1Q  Median      3Q     Max 
## -9.5903 -2.1565 -0.1169  1.8690 13.0604 
## 
## Coefficients:
##                Estimate Std. Error t value Pr(>|t|)    
## (Intercept)  -17.218435   4.644294  -3.707  0.00024 ***
## cylinders     -0.493376   0.323282  -1.526  0.12780    
## displacement   0.019896   0.007515   2.647  0.00844 ** 
## horsepower    -0.016951   0.013787  -1.230  0.21963    
## weight        -0.006474   0.000652  -9.929  < 2e-16 ***
## acceleration   0.080576   0.098845   0.815  0.41548    
## year           0.750773   0.050973  14.729  < 2e-16 ***
## origin         1.426141   0.278136   5.127 4.67e-07 ***
## ---
## Signif. codes:  0 '***' 0.001 '**' 0.01 '*' 0.05 '.' 0.1 ' ' 1
## 
## Residual standard error: 3.328 on 384 degrees of freedom
## Multiple R-squared:  0.8215, Adjusted R-squared:  0.8182 
## F-statistic: 252.4 on 7 and 384 DF,  p-value: < 2.2e-16
\end{verbatim}

\begin{Shaded}
\begin{Highlighting}[]
\KeywordTok{plot}\NormalTok{(autodata}\OperatorTok{$}\NormalTok{mpg,autodata}\OperatorTok{$}\NormalTok{year,}\DataTypeTok{main=}\StringTok{'Scatter Plot of X and Y'}\NormalTok{,}
     \DataTypeTok{xlab=}\StringTok{'MPG'}\NormalTok{,}\DataTypeTok{ylab=}\StringTok{'Production Year'}\NormalTok{)}
\KeywordTok{abline}\NormalTok{(fits)}
\end{Highlighting}
\end{Shaded}

\begin{verbatim}
## Warning in abline(fits): only using the first two of 8 regression coefficients
\end{verbatim}

\includegraphics{assignment1_files/figure-latex/unnamed-chunk-13-1.pdf}
None of the predictors seem to relate too much to the mpg feature. The
coefficient for the \textit{year} category (around 0.75) indicates that
the y-intercept of the best fit line crosses the y-axis at the year
1975. This would indicate that Cars made before 1975 would have a
negative horsepower - which is obviously not the case.

\begin{enumerate}
\def\labelenumi{\arabic{enumi}.}
\setcounter{enumi}{5}
\tightlist
\item
  Use the symbols \texttt{*} and \texttt{:} to fit linear regression
  models with interaction effects. What do you observe?\\
\end{enumerate}

\begin{Shaded}
\begin{Highlighting}[]
\NormalTok{fits2 <-}\StringTok{ }\KeywordTok{lm}\NormalTok{(mpg }\OperatorTok{~}\StringTok{ }\NormalTok{cylinders}\OperatorTok{:}\NormalTok{displacement}\OperatorTok{:}\NormalTok{horsepower}\OperatorTok{:}\NormalTok{weight}\OperatorTok{:}\NormalTok{acceleration}\OperatorTok{:}\NormalTok{year}\OperatorTok{:}\NormalTok{origin ,}\DataTypeTok{data=}\NormalTok{autodata)}
\KeywordTok{summary}\NormalTok{(fits2)}
\end{Highlighting}
\end{Shaded}

\begin{verbatim}
## 
## Call:
## lm(formula = mpg ~ cylinders:displacement:horsepower:weight:acceleration:year:origin, 
##     data = autodata)
## 
## Residuals:
##      Min       1Q   Median       3Q      Max 
## -10.0525  -3.8068  -0.8598   3.3632  19.2309 
## 
## Coefficients:
##                                                                     Estimate
## (Intercept)                                                        2.902e+01
## cylinders:displacement:horsepower:weight:acceleration:year:origin -8.136e-12
##                                                                   Std. Error
## (Intercept)                                                        3.782e-01
## cylinders:displacement:horsepower:weight:acceleration:year:origin  3.866e-13
##                                                                   t value
## (Intercept)                                                         76.72
## cylinders:displacement:horsepower:weight:acceleration:year:origin  -21.05
##                                                                   Pr(>|t|)    
## (Intercept)                                                         <2e-16 ***
## cylinders:displacement:horsepower:weight:acceleration:year:origin   <2e-16 ***
## ---
## Signif. codes:  0 '***' 0.001 '**' 0.01 '*' 0.05 '.' 0.1 ' ' 1
## 
## Residual standard error: 5.347 on 390 degrees of freedom
## Multiple R-squared:  0.5318, Adjusted R-squared:  0.5306 
## F-statistic:   443 on 1 and 390 DF,  p-value: < 2.2e-16
\end{verbatim}

Once again, there seems to be no correction between MPG and the other
features, wven when we build regression models with interactions.

\begin{enumerate}
\def\labelenumi{\arabic{enumi}.}
\setcounter{enumi}{6}
\tightlist
\item
  Try a few different transformations of variables, such as \(\log(X)\),
  \(\sqrt{X}\), \(X^2\). What do you observe?
\end{enumerate}

\begin{Shaded}
\begin{Highlighting}[]
\NormalTok{fit4 <-}\StringTok{ }\KeywordTok{lm}\NormalTok{(mpg}\OperatorTok{~}\NormalTok{(weight)}\OperatorTok{**}\DecValTok{2}\NormalTok{,}\DataTypeTok{data=}\NormalTok{autodata)}
\KeywordTok{summary}\NormalTok{(fit4)}
\end{Highlighting}
\end{Shaded}

\begin{verbatim}
## 
## Call:
## lm(formula = mpg ~ (weight)^2, data = autodata)
## 
## Residuals:
##      Min       1Q   Median       3Q      Max 
## -11.9736  -2.7556  -0.3358   2.1379  16.5194 
## 
## Coefficients:
##              Estimate Std. Error t value Pr(>|t|)    
## (Intercept) 46.216524   0.798673   57.87   <2e-16 ***
## weight      -0.007647   0.000258  -29.64   <2e-16 ***
## ---
## Signif. codes:  0 '***' 0.001 '**' 0.01 '*' 0.05 '.' 0.1 ' ' 1
## 
## Residual standard error: 4.333 on 390 degrees of freedom
## Multiple R-squared:  0.6926, Adjusted R-squared:  0.6918 
## F-statistic: 878.8 on 1 and 390 DF,  p-value: < 2.2e-16
\end{verbatim}

\hypertarget{problem-3-25}{%
\subsection{Problem 3 {[}25\%{]}}\label{problem-3-25}}

Using equation (3.4) in ISL, argue that in the case of simple linear
regression, the least squares line always passes through the point
\((\bar{x}, \bar{y})\).\\

Equation (3.4) in ISL:
\[\hat{\beta}_1 = \frac{\sum_{i=1}^n (x_i - \bar{x})(y_i-\bar{y})}
{\sum_{i=1}^n(x_i-\bar{x})^2}\] and
\[\hat{\beta}_0 = \bar{y} - \hat{\beta}_1\bar{x}\] Simple linear
regression produces a `line of best fit' using the two coefficients,
\(\beta_0\) and \(\beta_1\) as defined above. This fit takes the form of
\(f(X) = \beta_0 + \beta_1 X + \epsilon\). Given the difinition of
\(\beta_0\), we can rewrite this fit as:
\(f(X) = \bar{y} - \hat{\beta}_1(\bar{X}+X)\). Considering the defintion
of \(\bar{X}\) and \(\bar{y}\) (as averages), the subsequenct math then
follows that the regression will pass through some point given by
\(\bar{x} = \frac{1}{n}\sum_{i=1}^nx_i\) and
\(\bar{y} = \frac{1}{n}\sum_{i=1}^ny_i\), or some
simply,\((\bar{x}, \bar{y})\).

\hypertarget{problem-4-25}{%
\subsection{Problem 4 {[}25\%{]}}\label{problem-4-25}}

It is claimed in the ISL book that in the case of simple linear
regression of \(Y\) onto \(X\), the \(R^2\) statistic (3.17) is equal to
the square of the correlation between \(X\) and \(Y\) (3.18). Prove that
this is the case. For simplicity, you may assume that
\(\bar{x} = \bar{y} = 0\).\\
The \(R^2\) metric is defined as (James,70-3.17):
\[ R^2 \equiv 1 - \frac{RSS}{TSS}\] Where RSS is the
\textit{residual sum of squares} and TSS is the
\textit{total sum of squares}. The correlation coefficient is defined
as: \[ Cor(X,Y) = \frac{\sum_{i=1}^n(x_i-\bar{x})(y_i-\bar{y})}
{\sqrt{\sum_{i=1}^n(x_i-\bar{x})^2}\sqrt{\sum_{i=1}^n(y_i-\bar{y})^2}}\]

Thus if we take \(\hat{x} = \hat{y} = 0\) and square the function, we
arrive at: \[ r^2 = \big(Cor(X,Y)\big)^2 =
\frac{\big(\sum_{i=1}^nx_iy_i\big)^2}
{\sum_{i=1}^n(x_i)^2\sum_{i=1}^n(y_i)^2}\]

Thus, the RSS metric becomes \(\sum_{i=1}^n(y_i - \hat{y}_i)^2\) and the
TSS metric becomes \(\sum_{i=1}^n(y_i)^2\). This allows for the
insertion of RSS and TSS into the \((Cor(X,Y))^2\) function and thus we
can see that \(r^2\) and \(R^2\) are identical measurements.

\hypertarget{references}{%
\subsection{References}\label{references}}

Each reference is a link. Please open the PDF in a viewer if it is not
working on the website.

\begin{enumerate}
\def\labelenumi{\arabic{enumi}.}
\tightlist
\item
  \href{https://github.com/rstudio/cheatsheets/raw/master/data-visualization-2.1.pdf}{R
  GGPlot cheat sheet}
\item
  \href{https://machinelearningmastery.com/visualize-machine-learning-data-python-pandas/}{Python
  Pandas data visualization}
\item
  \href{https://r4ds.had.co.nz/index.html}{R For Data Science}
\item
  \href{https://www.rstudio.com/resources/cheatsheets/}{Cheatsheets}
  fffTheThe
\end{enumerate}

\end{document}
